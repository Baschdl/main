\section{Interaktion der Komponenten}
\textcolor{blue}{\textit{Die Interaktion zwischen den Komponenten während der Ausführung eines Use Case wird durch Sequenzdiagramme (oder Interaktionsdiagramme) auf Instanzen der Komponenten (repräsentiert durch Instanzen von Schnittstellenklassen/Interfaces) beschrieben. Die dabei ver-wendeten Operationen müssen sich später in den Schnittstellen der Komponenten wiederfinden.\\\\Es müssen nicht alle Use Cases hier beschrieben werden. Stattdessen sollen nur wesentliche oder charakteristische  Use Cases  (z. B. solche, deren Ablauf bei anderen Use Cases sehr ähn-lich stattfindet) beschrieben werden.}}\\
\\
\textbf{Interaktion bei Ausführung von\\
$<$Use Cae-ID$>$: $<$Use Case-Name$>$}

\begin{figure}[H]
\centering
\includegraphics[width=0.75\textwidth]{img/SequenzDiagrammInteraktion.png}
\caption{\textcolor{blue}{Durch eigenes Sequenzdiagramm ersetzen}}
\label{SequenzDiagrammInteraktion}
\end{figure}
